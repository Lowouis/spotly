\documentclass{article}
\usepackage[margin=2.5cm]{geometry}
\usepackage[T1]{fontenc}
\usepackage[utf8]{inputenc}
\usepackage[french]{babel}
\usepackage{enumitem}
\usepackage{hyperref}
\usepackage{titling}
\hypersetup{colorlinks=true, linkcolor=black, urlcolor=blue}

\setlength{\parskip}{6pt}
\setlength{\parindent}{0pt}

\title{Guide administrateur — Tâche cron Spotly}
\date{}

\begin{document}
\maketitle

\section*{Emplacement}
\begin{itemize}[leftmargin=1.2em]
  \item Fichier cron : \texttt{scripts/cron.mjs}
  \item Fichiers de logs : \texttt{logs/cron-yyyy-mm-dd.log.}
  \item Config PM2 : \texttt{ecosystem.config.cjs} (\texttt{spotly-cron})
\end{itemize}

\section*{Planification \& démarrage}
\begin{itemize}[leftmargin=1.2em]
  \item Fréquence : toutes les 30 minutes (\texttt{*/30 * * * *})
  \item Démarrage/reload (sur le serveur) :
\begin{verbatim}
pm2 reload ecosystem.config.cjs --only spotly-cron --update-env
pm2 logs spotly-cron --lines 100
\end{verbatim}
\end{itemize}

\section*{Variables d'environnement}
\begin{itemize}[leftmargin=1.2em]
  \item \textbf{RUN\_AT\_START=1} : lance un cycle immédiatement au démarrage, puis selon la planification.
  \item \textbf{RUN\_ONCE=1} : exécute un cycle unique puis quitte (test manuel).
  \item \textbf{LOGS\_DIR} : répertoire des logs (optionnel). Par défaut \texttt{logs/} à la racine du projet.
\end{itemize}

\section*{Logs}
\begin{itemize}[leftmargin=1.2em]
  \item Dossier : \texttt{logs/} (ou la valeur de \texttt{LOGS\_DIR} dans le .env)
  \item Fichiers : \texttt{cron-YYYY-MM-DD.log} (un par jour)
  \item Consultation :
\begin{verbatim}
tail -n 200 "${LOGS_DIR:-logs}/cron-$(date +%F).log"
pm2 logs spotly-cron --lines 200
\end{verbatim}
\end{itemize}

\section*{Ce que fait chaque cycle}
\begin{itemize}[leftmargin=1.2em]
  \item \textbf{Rattrapage (sécurité)} si la cron a été interrompue :
  \begin{itemize}
    \item Passe en \texttt{USED} les réservations \texttt{ACCEPTED} déjà commencées (auto-pickup).
    \item Passe en \texttt{ENDED} + \texttt{returned=true} les \texttt{ACCEPTED} terminées (profil \texttt{FLUENT}).
    \item Logue les \texttt{ACCEPTED} expirées non auto-terminables (ex. \texttt{HIGH\_TRUST}) pour traitement manuel.
  \end{itemize}
  \item \textbf{Auto-pickup} : \texttt{ACCEPTED} \(\rightarrow\) \texttt{USED} si période en cours (\texttt{FLUENT}/\texttt{HIGH\_TRUST}, priorité ressource > catégorie > domaine).
  \item \textbf{Auto-return} : \texttt{USED} \(\rightarrow\) \texttt{ENDED} + \texttt{returned=true} si fin dépassée (\texttt{FLUENT}, même priorité).
  \item \textbf{Info retard} : logue le nombre de réservations encore en \texttt{USED} dont la fin est dépassée.
\end{itemize}


\section*{Tests rapides}
\begin{itemize}[leftmargin=1.2em]
  \item Un cycle immédiat (unique) :
\begin{verbatim}
RUN_ONCE=1 node scripts/cron.mjs
\end{verbatim}
  \item Immédiat + planification 30 min :
\begin{verbatim}
RUN_AT_START=1 node scripts/cron.mjs
\end{verbatim}
\end{itemize}

\section*{Dépannage}
\begin{itemize}[leftmargin=1.2em]
  \item \textbf{Pas de logs} : vérifier \texttt{pm2 logs}, droits d'écriture du dossier, valeur de \texttt{LOGS\_DIR}, app \texttt{spotly-cron} en ligne.
  \item \textbf{Pas d'exécution au démarrage} : s'assurer que \texttt{RUN\_AT\_START=1} est bien défini dans PM2 et recharger avec \texttt{--update-env}.
  \item \textbf{Fuseau horaire} : les timestamps sont ceux du serveur.
\end{itemize}

\section*{Dictionnaire}
\begin{itemize}[leftmargin=1.2em]
    \item \textbf{USED} : \texttt{UTILISÉ}
    \item \textbf{ACCEPTED} : \texttt{ACCEPTÉ}
    \item \textbf{ENDED} : \texttt{TERMINÉ}
    \item \textbf{FLUENT} : \texttt{LIBRE}
    \item \textbf{HIGH\_TRUST} : \texttt{PAR CLIQUE DE RESTITUTION}
\end{itemize}

\end{document}